\documentclass[letterpaper,10pt]{article}
\usepackage{hyperref}
\hypersetup{
    colorlinks=true,
    linkcolor=blue,
    filecolor=magenta,      
    urlcolor=blue,
}
\urlstyle{same}
\setlength{\topmargin}{-.5in}
\setlength{\textheight}{10in}
\setlength{\oddsidemargin}{-.25in}
\setlength{\textwidth}{7in}
\pagestyle{empty}

\begin{document}

\begin{center}

\tabcolsep=5pt
{\LARGE \textbf{Math 18C\hspace{.25in} Multivariable Calculus  \hspace{.25in}Fall 2018}} \\
\vspace{.2in} \textit{ }
{\bf Sec. 3302,  \ \ MWF 11:20am - 12:55 pm, Room: Sequoia 102}
\rule{8cm}{0.4pt}
\end{center}

%\setlength{\unitlength}{1in}



 

%\renewcommand{\arraystretch}{2}

\vskip.25in
\noindent\textbf{Instructor:} Joe Manlove, Sequoia 103, manlovej@yosemite.edu, Office: 209.588.5083, Cell: 406.600.7281
\vskip.25in
\noindent\textbf{Office Hours:} 10:00 to 11:00 MWF, 11:00 to 1:00 Tuesday, and 12:30 to 1:00 Thursday (or by appointment)
\vskip.25in
\noindent\textbf{Contact:} To contact me, try my office first. I am often in my office outside of my hours. Feel free to come see me anytime. If you can't locate me, feel free to email or text me. I will respond quickly during reasonable hours and somewhat more slowly if I am asleep.

\vskip.25in
\noindent\textbf{Accessibility:} Columbia College is committed to providing access and inclusion for all persons with disabilities. Students with verified disabilities who are registered with Columbia College’s Disabled Student Programs and Services (DSPS) who need specific access in this course, such as accommodations, should contact the instructor early in the semester so that accommodations may be implemented as soon as possible. Students can connect with Disabled Student Programs and Services (DSPS), located in upper Manzanita or call 209-588-5130 for an appointment with the DSPS Coordinator/Counselor. More information is available at \href{https://www.gocolumbia.edu/dsps/}{https://www.gocolumbia.edu/dsps/}. Through DSPS, course Accessibility Forms, the Academic Accommodation Plan (AAP) and Letter of Accommodation (LOA) may be created and brought or sent to instructors. This process informs instructors of potential access and accommodations that are reasonable. This syllabus is available in alternate formats upon request.

\vskip.25in
\noindent\textbf{Textbook:} \href{https://www.whitman.edu/mathematics/multivariable/multivariable.pdf}{The textbook is here.} It's a free pdf, you can order a printed copy from Lulu.




\vskip.25in
\noindent\textbf{Prerequisites:}
C or better in Math 18B 

\vspace*{.15in}

\noindent\textbf{Grades:}
\begin{center}

\begin{tabular}{|l|l|l|l|l|l|l|l|l|l|l|l|}
    \hline
    Grade & A & B&C&D&F\\
    \hline
    Percentage& 90-100 & 80-89& 70-79&  60-69& 0-59\\
    \hline
\end{tabular}
\vskip.25in
\begin{tabular}{|l|l|l|l|l|l|}
\hline
 Videos&Quizzes&Test 1&Test 2& Project &Final\\
 \hline
 50&50&100 &100 &50 &150\\
 \hline
\end{tabular}
\end{center}

\vskip.25in
\noindent\textbf{Videos}: You'll need to do the boring bits of class on your own outside of class, so I need you to \href{https://edpuzzle.com/join/dugiwec}{go to edpuzzle} and sign up for my class. The boring bits are the part where I talk at you, the exciting bits are where your brain meets the paper. This means you probably will be able to get mostly or entirely done with the homework in class.


\vskip.25in
\noindent\textbf{Homework}: If you have questions (or would like feedback on homework) please come to my office hours, email me, go to the Math Lab, or check out \href{https://www.youtube.com/playlist?list=PLL_y29hK_RhSAPy7OIKs5xmj0_9OPsI23}{the videos on the 18C playlist on the Youtube Channel}.

\vskip.25in
\noindent\textbf{Quizzes}: Quizzes will be given daily in class. There will be approximately 40 quizzes over the semester; 35 of them will count towards your grade.

\vskip.25in
\noindent \textbf{Topics Covered}:  Vectors and solid analytic geometry, vector valued functions, partial differentiation, multiple integrals, vector fields and vector calculus.


\clearpage

\noindent\textbf{Outcomes:}\\The Math Department SLOs state students successfully completing this course will be able to:
\begin{itemize}
\item Use vectors in multidimensional space
\item Apply derivatives and integrals in multidimensional space
\item Apply Stokes Theorem
\end{itemize}
The Course Objectives from the Course Outline of Record:
\begin{itemize}
\item Find vector, parametric, and scalar equations of lines and planes in
three-dimensional space
\item Use vector operations (dot product, cross product, triple product, and projections)
to compute distances, angles, areas, and volumes in three dimensions.
\item Find the limit of a function of several variables at a point.
\item Determine differentiability. If applicable, find and interpret partial derivatives,
directional derivatives, higher derivatives, and gradient vector fields for functions of
several variables.
\item Apply derivatives and integrals to problems of position, velocity, acceleration, and
arc length.
\item Correctly apply the chain rule for transformations.
\item Find potential extrema and saddle points. Test to determine if such a point is a
maximum, minimum, or saddle point.
\item Solve optimization problems and constraint problems using either extrema or
Lagrange mulitpliers.
\item Set up and evaluate multiple integrals in rectangular, cylindrical, and spherical
coordinates to find volume, mass, and surface area.
\item Set up and evaluate line integrals.
\item Find the divergence and curl of a vector field.
\item Set up surface integrals and apply Green's theorem, the divergence theorem and
Stokes' theorem. 
\end{itemize}




\clearpage


\begin{center}
This schedule is approximate, be prepared for small adjustments to it.
\begin{tabular}{|c|c|c|} \hline
\hspace{1.5in}  & \hspace{1.5in} & \hspace{1.5in} \\
\bf{Monday}     & \bf{Wednesday} & \bf{Friday}  \\
                            &                &       \\ \hline \hline
% Week 1
 \bf{August} \hfill\bf{27}& \hfill\bf{29}&\hfill\bf{31} \\
 Syllabus & 12.2   &  12.3  \\
 Latex &     &  12.4   \\ \hline

% Week 2
 \bf{September} \hfill \hfill\bf{3} &\hfill\bf{5} & \hfill\bf{7} \\
 Labor Day &  12.5  &  13.1   \\
 No Class &     &   \\ \hline

% Week 3
 \hfill\bf{10} & \hfill\bf{12} & \hfill\bf{14} \\
 13.2 & 13.3 &   13.4    \\
  &  &       \\ \hline
% Week 4
  \hfill\bf{17} &\hfill\bf{19} & \hfill\bf{21} \\
 14.1  & 14.2 &    14.3   \\
  &  &       \\ \hline
% Week 5
 \hfill\bf{24} & \hfill\bf{26} & \hfill\bf{28} \\
   14.4   &   14.5 & 14.6 \\
  &     &     \\ \hline

% Week 6
\bf{October}\hfill\bf{1} & \hfill\bf{3} & \hfill\bf{5} \\
  Review & Test 1 &    14.7   \\
  &  &       \\ \hline
% Week 7
 \hfill\bf{8} &\hfill\bf{10} & \hfill\bf{12} \\
  14.7 & 14.8 &    14.8   \\
  &  &       \\ \hline
  
% Week 8
 \hfill \bf{15} & \hfill\bf{17} & \hfill\bf{19} \\
  15.1 & 15.2 &   15.3    \\
  &  &       \\ \hline
% Week 9
\hfill\bf{22} &  \hfill\bf{24} & \hfill\bf{26} \\
 15.4 & 15.5 &   15.6    \\
  &  &       \\ \hline
%Week 10
 \hfill\bf{29} & \hfill \bf{31} &\bf{November} \hfill\bf{2} \\
 15.6 & 15.7 &   Review    \\
  &  &       \\ \hline
% Week 11
 \hfill\bf{5} &\hfill\bf{7} & \hfill\bf{9} \\
 Test 2 &  16.1   & 16.2  \\
  &     &  16.3  \\ \hline
% Week 12
 \hfill\bf{12} &\hfill\bf{14} & \hfill\bf{16} \\
  Veteran's & 16.4 &  16.5     \\
  Day &  &       \\ \hline
% Week 13
 \hfill\bf{19} & \hfill\bf{21} & \hfill\bf{23} \\
 16.6 & 16.7    &  Thanksgiving   \\
  &     &  No Class   \\ \hline
% Week 14
 \hfill\bf{26} & \hfill\bf{28} &\hfill\bf{30} \\
  16.8 & 16.8 &  16.9     \\
  &  &       \\ \hline
            
% Week 15
 \bf{December} \hfill\bf{3} &\hfill\bf{5} & \hfill\bf{7} \\
16.9 &  Review   &  Review   \\
  &     &     \\ \hline
              



\end{tabular}
\end{center}
\begin{center} {\bf Final Exam} :: Monday, December 10 :: 11am - 1pm\end{center}



\end{document}




